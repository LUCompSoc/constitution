\documentclass[parskip=half]{scrartcl}

\usepackage{blindtext}
\usepackage{hyperref}

\hypersetup{
    colorlinks=true,
    linkcolor=blue,
    filecolor=magenta,
    urlcolor=cyan,
    pdftitle={The Computer Science Society of Lancaster University - Constitution},
    pdfpagemode={UseOutlines},
    breaklinks=true,
}

\urlstyle{same}

\title{The Computer Science Society of\\ Lancaster University - Constitution}
\author{The Compsoc Executive of 2024/25}
\date{June 2024}


\begin{document}
    \maketitle

    \clearpage
    \tableofcontents

    \clearpage
    \section{Preface}
        \label{preface}
        \subsection{Format}
            \label{preface--format}
            All sections of this document are to be considered normative unless otherwise specified.

        \subsection{Definitions}
            \label{preface--definition}
            This section is non-normative.

            \begin{enumerate}
                \item The "AGM": the Annual General Meeting.
                \item An "EGM": an Extraordinary General Meeting.
                \item An “Extreme Circumstance”: A situation where the collapse of the society is imminent within fourteen (14) days, if left with no intervention.
                \item The "Executive": the Executive Committee of the society.
                \item The "Membership": the collection of persons holding membership of the Society in any kind.
                \item "SCC": \href{https://www.lancaster.ac.uk/scc}{Lancaster University School of Computing and Communications}.
                \item The "SU", or the "Union": \href{https://www.lancastersu.co.uk}{Lancaster University Student's Union}.
                \item The "University": \href{https://www.lancaster.ac.uk}{Lancaster University}.
                \item A "Student": An Undergraduate or Postgraduate registered student at the University.
                \item A "Qualified Majority": A requirement for a vote to be successful, where greater than fifty percent (50\%) of the unified vote is needed, unless otherwise specified.
            \end{enumerate}

    \clearpage
    \section{Documentation}
        \label{documentation}

\begin{enumerate}
    \item This and all other documents produced by the society are superseded by applicable law.

    \item This and all other documents produced by the society are superseded by Union documents on Governance and Bye-laws, in addition to any and all other documents pertaining to the governance and regulation of societies.

    \item This and all other documents produced by the society are superseded by SCC policy, organisation, and rules.
    
    \item In the event of a contradiction between the society, its documentation and or other documentation and bodies, the applicable laws and local governance takes precedence and advice should be sought by an appropriate party. 

    \item Post advice, any affected document/s belonging to the society shall be updated. All members need to be informed of the update.

\item Information on documentation amendments can be found in section 6.2 Documentation Amendments

\end{enumerate}

    \clearpage
    \section{General Status}
        \label{general}
        \subsection{Identification}
            \label{general--identification}
            \begin{enumerate}
                \item The name of this society shall be ”The Computer Science Society of Lancaster University”.
                \item This society may be referred to either as "LUCompSoc" or simply "CompSoc".
                \item This society is categorised as an Academic Society.
            \end{enumerate}

        \subsection{Governance}
            \label{general--governance}
            \begin{enumerate}
                \item The student body of this society is governed jointly by the Union, SCC, and the Executive (\ref{executive}).
            \end{enumerate}

    \clearpage
    \section{Membership}
        \label{membership}
        \subsection{Qualification}
            \label{membership--qualification}
            \begin{enumerate}
                \item Membership of the society is available to all persons via a membership fee.
                
                \item Persons in extreme circumstances who are unable to pay the membership fee may consult with both the President and the Treasurer, who together, will decide on fee exemption on a case by case basis and at their own unanimous discretion.

                \item Membership of the society may either be Full or Honorary.
                
                \item Honorary membership may be granted either for life or for a set period of time, subject to the agreement of a qualified majority at a General Meeting. This privilege shall be granted normally to former members for outstanding service to the society.
                
                \item The individual has the right to refuse honorary membership.
                
                \item Honorary members shall never be obligated to pay a membership fee, however, may do so if they desire. Choosing to pay for membership does not discount the Honorary status.
                
                \item Membership status may be revoked by qualifying vote at a general meeting if the member directly or indirectly negatively impacts the appearance or function of the society following a finding of such potential impact to the society by the executive.
            \end{enumerate}
            
        \subsection{Rights}
            \label{membership--rights}
            All members may:
            \begin{enumerate}
                \item Attend and speak at general meetings.
                \item Participate in society activities, subject to their individual requirements.
                \item Propose a motion to be discussed at general meetings.
                \item Vote at general meetings.
                \item Have the right to nominate or second a candidate in elections.
            \end{enumerate}

            Additionally, members that are students at Lancaster University:
            \begin{enumerate}
                \item Have the right to stand for all positions in elections, Unless they expect not to be a student at any point during their tenure.
            \end{enumerate}
            
        \subsection{Exclusion}
            \label{membership--exclusion}
            \begin{enumerate}
                \item The President reserves the right to suspend membership of an individual for:
            \begin{enumerate}
                \item Breaching this, the Constitution,
                \item Breaching the Safety Framework of the Union,
                \item Bringing the society into disrepute,
                \item Bullying, harassment, and all other forms of distress are imposed on another individual member of society or on the society as a whole.
                \item Any other matter which would be deemed dangerous to the society, subject to the Complaints and Appeals procedure (\ref{complaints-appeals}). 
            \end{enumerate}

            \item On suspension of a member, they will be formally notified by the executive.

            \item A statement can be prepared by the member to present before the entire executive committee within 7 days of the suspension.
            
            \item If a statement is not submitted within 7 days, it is assumed that a blank statement has been provided.

            \item The executive will then have up to 7 days to review the statement and hold a vote, whereby membership may be reinstated in full, refused, revoked or permanently banned with the approval of the entire executive committee.
            
            \item If the vote succeeds, the membership fee will not be reimbursed.

            \item If the vote fails, or the vote does not take place within 7 days of each executive being available from the date the statement is submitted, the suspension shall be lifted.
 
            \item If the individual is a member of the executive committee, please refer to the guide for impeachment. (\ref{executive--impeachment})
        \end{enumerate}
        
        \subsection{Extreme Circumstances}
            \label{membership--extreme-circumstances}
            In the event of an extreme circumstance, the executive may perform actions which are not currently covered by the documentation of the society and may or may not immediately inform the membership, until the circumstances stabilise.

            In the event of potential danger towards the society in a time frame greater than that defined by an extreme circumstance, an EGM shall be called and the membership must be informed.

    \clearpage
    \section{The Executive Committee}
        \label{executive}
        \subsection{Description}
        \label{executive--description}
        \begin{enumerate}
            \item The Executive is the primary governing body of the Society.
            \item The Executive is deemed to be in session during the entirety of their tenure.
            \item No member may be compelled to perform duties of any kind.
        \end{enumerate}
        
        \subsection{Executive Positions}
            \label{executive--executive-positions}
            \begin{enumerate}
                \item All members holding positions shall be a full member of the Society and a current student at Lancaster University.

                \item All members holding positions shall serve as ambassadors for the Society. This is to include, but not be limited to, the initiation of relationships with and coordination with current and future Industry Contacts.

                \item All available positions shall be elected in General Meetings by a Qualified Majority of the Membership.

                \item The members holding positions shall have the right to vote in Executive Meetings.

            \end{enumerate}
            
            \subsubsection{President}
                \label{executive--executive-positions--president}
                \begin{enumerate}
                    \item The President shall be the primary representative of the society to the Union, to the University, and to external bodies.
                    
                    \item The President shall fulfil all the duties demanded of them by the Union, the University, the Society and external bodies.
                    
                    \item The President shall coordinate and oversee the activities both of the Executive and of the society as a whole.
                    
                    \item The President is empowered to delegate any and all duties and responsibilities to other members of the Executive.
                    
                    \item The President is empowered to make decisions on behalf of the Executive.
                    
                    \item The Executive may call the President to account for any decisions made on its behalf.
                    
                    \item The President shall be responsible for maintaining the reputation of the society, the members of the society and the possessions of the society.

                    \item The President shall act as Chair to all society meetings.
                   
                    \item The President shall act as Returning Officer at all elections, acting as a guardian of the democratic process, and assisting in the peaceful transition of power.
                    
                    \item No individual can hold the position of President more than once.
                \end{enumerate}
            
            \subsubsection{Vice-President}
                \label{executive--executive-positions--vice-president}
                \begin{enumerate}
                    \item The Vice-President shall assist the President in the completion of Presidential duties.
                    
                    \item The Vice-President shall aid all other officers to ensure the  integrity, security and promotion of the society, as well as the wellbeing of all members. 
                    
                    \item In the absence of the President, the Vice-President shall speak with Presidential authority for matters which do not impact the macro goals of the society.
                \end{enumerate}
            
            \subsubsection{Secretary}
                \label{executive--executive-positions--secretary}
                \begin{enumerate}
                    \item The Secretary shall be jointly responsible with the Treasurer, President and Vice-President for the administration of all manners relating to memberships and subscriptions.
                    
                    \item The Secretary shall be responsible for recording  minutes at meetings (or their equivalent) and their distribution.
                    
                    \item The Secretary shall be responsible for ensuring any and all public documentation is accurate and current.
                \end{enumerate}
            
            \subsubsection{Treasurer}
                \label{executive--executive-positions--treasurer}
                \begin{enumerate}
                    \item The Treasurer shall maintain the society finances to be transparent and well-ordered, as well as be involved with all revenue building methods and strategies.
                    
                    \item The Treasurer shall ensure that all financial regulations, of the Union, of the University, and of applicable law, are adhered to.
                    
                    \item The Treasurer shall advise the Executive as to the finances of the society.
                    
                    \item The Treasurer shall present a Statement of Revenue and Expenditure to the assembly present at the AGM, and there it shall be ratified.
                    
                    \item The Treasurer shall, with the Secretary, President and Vice-President, be jointly responsible for the administration of all manners relating to memberships and subscriptions.
                \end{enumerate}
            
            \subsubsection{Education Officer}
                \label{executive--executive-positions--education-officer}
                \begin{enumerate}
                    \item The Education Officer shall be jointly responsible with the President and Vice President for the organisation of Educational Events.
                \end{enumerate}
            
            \subsubsection{Events Officer}
                \label{executive--executive-positions--events-officer}
                \begin{enumerate}
                    \item The Events Officer shall be responsible for the organisation of large-scale events, in coordination with other applicable members of the Executive.
                    
                    \item The Events Officer shall be responsible for the creation of success criteria for each event.
                    
                    \item The Events Officer shall be empowered to deputise any member as is seen to be appropriate to ensure the success of the event.
                \end{enumerate}
            
            \subsubsection{Publicity Officer}
                \label{executive--executive-positions--publicity-officer}
                \begin{enumerate}
                    \item The Publicity Officer shall be responsible for the publication and proclamation of the society and its events.
                    
                    \item The Publicity Officer shall be responsible for the upkeep of the societies presence on all online platforms and social networks utilised by the Society.
                    
                    \item The Publicity Officer shall be responsible for the production and organisation of all marketing materials for the promotion of the Society, its events and its campaigns.
                    
                    \item The Publicity Officer shall be jointly responsible with the Treasurer and the President for the maintenance and procurement of publicity supplies.
                    
                    \item The Publicity Officer shall be responsible for the maintenance of documentation pertaining to publicity materials, including, but not being limited to:
                    \begin{itemize}
                        \item stylistic materials 
                        \item templates
                        \item physical materials.
                    \end{itemize}
                \end{enumerate}
            
            \subsubsection{Technical Officer}
                \label{executive--executive-positions--technical-officer}
                \begin{enumerate}
                    \item The Technical Officer shall be responsible for the coordination, maintenance, and reliability of all technical systems and equipment utilised by the Society
                    
                    \item The Technical Officer shall, in consultation with the Treasurer, be responsible for the procurement of equipment utilised by the Society.
                    
                    \item The Technical Officer shall be responsible for the oversight of and provision of resources to projects undertaken by the Society.
                    
                    \item The Technical Officer shall be responsible for the completion of a Feasibility Analysis of project proposals and shall advise the Executive as to the feasibility of project proposals.
                \end{enumerate}
            
            \subsubsection{Wellbeing Officer}
                \label{executive--executive-positions--wellbeing-officer}
                \begin{enumerate}
                    \item The Wellbeing Officer shall be responsible for the wellbeing and welfare of the Membership.
                    
                    \item The Wellbeing Officer shall ensure all Society activities conform both to the Safety Framework of the Union.
                \end{enumerate}
            
            \subsubsection{Security Officer}
                \label{executive--executive-positions--security-officer}
                \begin{enumerate}
                    \item The Security Officer shall be responsible for the digital and physical safety of the society's possessions and inventory.
                    
                    \item The Security Officer shall be jointly responsible with the Treasurer and the President for the acquisition and implementation of additional security measures.
                    
                    \item The Security Officer shall, in consultation with the Technical Officer, be responsible for the security of all technical systems utilised by the Society.
                    
                    \item The Security Officer shall report security vulnerabilities to the executive committee, and shall devise, advise and execute plans to patch the discovered vulnerability. This includes vulnerabilities discovered by the security officer, as well as any individual that has reported a security vulnerability.
                    
                    \item The Security Officer shall be jointly responsible with the Treasurer and the President to promote penetration testing and bug bounties, for the education of the membership, as well as improvement of the societies security practices.
                \end{enumerate}
            
            \subsubsection{Audio Visual Officer}
                \label{executive--executive-positions--audiovisual-officer}
                \begin{enumerate}
                    \item The Audio Visual Officer shall be responsible for the photography, videography and audio works for all society activities.

                    \item The Audio Visual Officer shall be jointly responsible with the Treasurer and the President to procure new Audio/Visual equipment.
                \end{enumerate}
            
            \subsubsection{Careers Officer}
                \label{executive--executive-positions--careers-officer}
                \begin{enumerate}
                    \item The Careers Officer shall be responsible for events catered towards the career development of the membership.
                \end{enumerate}

        \subsection{Non-Executive Positions}
            \label{executive--non-executive-positions}
            \begin{enumerate}
                \item Each office may have additional, non-executive, sub-positions.
                \item Any member holding a sub-position shall be a full member of the society and a current student at Lancaster University.
                \item Each member of the extended category shall be the manager of their elected office and the sub-positions within it.
                \item Sub-positions can be opened at any time and are available for all members to apply for.
                \item The success of an applicant for a sub-position is subject to the approval of the respective office manager and a qualified majority of the executive committee.
                \item Sub-positions shall focus on improving the society by assisting the office of their application.
                \item The exact details of each sub-position shall be outlined by their respective office manager.
                \item If a member desires to aid the society via a sub-role, yet no such sub-role currently exists, the member is invited to speak with the President and the Office Manager, where a specific sub-position may be opened, subject to the approval of the respective office manager and a qualified majority of the executive committee.
                \item Sub-positions are non-executive positions, and thus, members with such roles do not have the right to vote in executive meetings. Members with sub-positions are still encouraged to provide feedback to the executive for the benefit of the society. Their right to vote in general meetings is not affected. 
                \item An individual may be released from their sub-position for any of the reasons mentioned in Exclusion (\ref{membership--exclusion}).

            \end{enumerate}

        \subsection{The Cabinet}
            \label{executive--cabinet}
            \begin{enumerate}
                \item Within the Executive shall be the Cabinet, divided into two categories, Core and Extended.
                \item Each member of the Extended category shall be the manager of their elected Office.
                \item Positions from within these Offices shall have delegated power by their superior Extended Cabinet member, and will be allocated to individuals by the Extended Cabinet member with the approval of the Core Cabinet.
                \item All members of the Cabinet shall attend and complete all training offered and required by the Activities Office of the Union.
            \end{enumerate}

            \subsubsection{The Core Cabinet}
                \label{executive--cabinet--core}
                
                The Core Cabinet contains the following positions
                \begin{itemize}
                    \item President (\ref{executive--executive-positions--president})
                    \item Vice-President (\ref{executive--executive-positions--vice-president})
                    \item Secretary (\ref{executive--executive-positions--secretary})
                    \item Treasurer (\ref{executive--executive-positions--treasurer})
                \end{itemize}
            
            \subsubsection{The Extended Cabinet}
                \label{executive--cabinet--extended}
                the Extended Cabinet contains the following positions
                \begin{itemize}
                    \item Education Officer (\ref{executive--executive-positions--education-officer})
                    \item Events Officer (\ref{executive--executive-positions--events-officer})
                    \item Publicity Officer (\ref{executive--executive-positions--publicity-officer})
                    \item Audio Visual Officer (\ref{executive--executive-positions--audiovisual-officer})
                    \item Technical Officer (\ref{executive--executive-positions--technical-officer})
                    \item Security Officer (\ref{executive--executive-positions--security-officer})
                    \item Careers Officer (\ref{executive--executive-positions--careers-officer})
                    \item Wellbeing Officer (\ref{executive--executive-positions--wellbeing-officer})
                \end{itemize}
                
        \subsection{Impeachment}
            \label{executive--impeachment}
            \begin{enumerate}
                \item Any member of the Cabinet may be impeached for (but not limited to):
                    \begin{enumerate}
                        \item Gross negligence of duty
                        
                        \item Bullying, abuse or harassment in any form towards any individuals or group

                        \item Serious breach of safety regulations
                        
                        \item Posing a grave threat to the safety and longevity of the Society or the Student Union
                        
                        \item Embezzlement of the resources of the Society
                        
                        \item Abuse of power of a role
                        
                        \item Abuse of financial power (e.g. refusal to allocate without valid reason)
                    \end{enumerate}
                \item An impeachment can be invoked by one of the following manners:
                    \begin{enumerate}
                        \item By a motion of impeachment from at least one member of the Core Cabinet
                        
                        \item By a joint motion from a majority of members of the Extended Cabinet
                        
                        \item A letter signed by 10 members of the Society and one member of the Cabinet delivered to the Secretary, Vice-President, or President.
                    \end{enumerate}
                \item Upon an impeachment motion being called, the following actions must occur for the impeachment to be valid:
                    \begin{enumerate}
                        \item There must be a meeting of the Cabinet within 7 days of receipt of a valid impeachment motion to set out a timeline for a vote to be held on the impeachment motion, this vote must occur within 14 days of this meeting and must be held online, with a minimum of 48 hours notice and a minimum online voting window of 3 hours
                        
                        \item The Cabinet must vote on whether to carry the motion within this meeting
                        
                        \item Within this meeting of the Cabinet, the subject of the impeachment notice must be given the opportunity to step down silently with no notice of impeachment being released to the membership

                        \item An announcement of the impeachment notice must be released following the Cabinet meeting via the official communication channels
                        
                        \item The subject of the impeachment motion must be given opportunity to lay out their defence to the membership in the form of ONE official communication through the official communication channels
                        
                        \item In the event of a Cabinet member responding to this official communication, the subject of the impeachment motion must be given the opportunity to lay out their rebuttal in the form of ONE official communication through the official communication channels
                        
                        \item A simple majority of voters within the impeachment vote must agree to impeachment
                    \end{enumerate}

                \item Upon the passing of a valid impeachment, a by-election will be triggered which must be held within 14 days, following the same voting rules as that of the impeachment, with the only exception being the vote threshold being set to a plurality for any given candidate.

                \item In the event of a Core Cabinet member being impeached, the Vice-President will take the impeached person’s role and position until a replacement is elected.
                
                \item In the case that the Core Cabinet member impeached is the Vice-President, no interim Vice-President will be selected.
                
                \item In the case of the position of Vice-President being vacant, a member of the Extended Cabinet will be voted on by the Cabinet to take on this interim position.
            \end{enumerate}
            
    \clearpage
    \section{General Meetings}
        \label{gm}
        \subsection{Assembly}
            \label{gm--assembly}
            \begin{enumerate}
                \item The Assembly at a General Meeting shall represent the Membership.
                    \subitem The voice of the Assembly shall thus constitute the ultimate governing institution of the Society.
                    
                \item All members of the Society shall have the right to be present and vote at all General Meetings.
                \item The executive shall take reasonable measures to ensure all members are aware of general meetings and when they will take place, by announcement on at least three of the official communication platforms of the society, as well as repeating said announcement  at the start and conclusion of all activities post announcement.
                \item Decisions made by the Assembly shall be binding to the Executive.
                \item In the event of an executive, or membership vote resulting in a stalemate, The President shall cast an additional vote to end the stalemate. This is the only time The President may vote more than once. Any suspicion of The President abusing this right, or employing this practice in a setting which is not to settle a stalemate should be reported through the complaints and appeals process.
            \end{enumerate}
        
        \subsection{Documentation  Amendments}
            \label{gm--documentation-amendment}
            \begin{enumerate}
                \item Amendments to the Constitution can be proposed by any member of the Society to the executive for advice and acceptance at least 14 days prior to a General Meeting.
                
                \item The proposer must show the executive how the update would not cause harm to the society.
                
                \item If upon reflection, the proposed amendment would not endanger the society, its membership, its possessions, or its affiliates, the executive shall publicly announce the proposed amendment to the membership for reflection between 7 and 14 days prior to the General Meeting.
                
                \item The proposer should also take reasonable measures to inform the membership of their proposed update, as well as show how the update would be of benefit to the society, provided it is agreed the update would not pose a risk to the society. 
                \item The executive may refuse a proposal, if its passing would endanger the society, its membership, its possessions or its affiliates.
                
                \item If an individual/s who proposed an amendment retracts the proposal prior to the General Meeting, the membership shall be informed immediately.
                
                \item Proposed amendments shall be ratified and enacted only with the approval of two-thirds, or higher, of the Assembly.

                \item When proposing an amendment, the following shall be included;
                \begin{enumerate}
                    \item Original clause (when available)
                    \item Affected document
                    \item Proposed clause
                    \item Description of the proposed amendment
                \end{enumerate}
            \end{enumerate}
        
        \subsection{Dissolution of the Society}
            \label{gm--disolution}
            \begin{enumerate}
                \item The Society may be dissolved at a General Meeting, provided that at least twenty-one (21) days notice of the intention of dissolution has been given to the Membership.
                \item A qualified majority of at least two-thirds of the highest between the Assembly or the average weekly attendance of when sessions are in motion is required for the Motion of Dissolution to be effective.

                \item In the event of the Dissolution of the Society, the remaining executive shall follow the guidance of the Student Union.
            \end{enumerate}
            
        \subsection{The Annual General Meeting}
            \label{gm--agm}
            \begin{enumerate}
                \item Exactly one AGM shall be held each Academic Year in the Lent Term.
                \item An AGM may be called by:
                    \begin{enumerate}
                        \item the President
                        \item simple majority of the core cabinet
                        \item simple majority of the extended cabinet
                        \item by written request of at least five (5) members of the Society, plus one member of the core cabinet or two members of the extended cabinet.
                    \end{enumerate}
                \item Any request to hold an AGM shall be given to an available member of the Cabinet. This request shall then be proposed to the Executive as soon as possible, who must then call an election as soon as possible.
                \item At the AGM, the following shall occur:
                    \begin{enumerate}
                        \item Any amendments to the Constitution of the Society shall be proposed and either ratified or discarded;
                        \item The President shall present a report of the activities of the Society for the prior year;
                        \item The Treasurer shall present a statement of accounts for ratification by the Assembly;
                        \item The Executive for the following year shall be elected, following the Election Procedure (\ref{bye-election--election-procedure}).

                    \end{enumerate}
                                        
                    \item At each AGM, all electable positions shall be deemed to be available for the following year. 

                        \subitem In the event a member of the Executive desires to remain in office, said member shall declare candidacy, even in the absence of opposition.
            \end{enumerate}
        
        \subsection{Extraordinary General Meetings}
            \label{gm--egm}
            \begin{enumerate}
                \item An EGM may be called for a number of reasons including, but not being limited to, invocation of the Articles of Impeachment or an exodus of the incumbent Executive.
                \item Once called, an EGM must be held within fourteen (14) days.
            \end{enumerate}
        
        \subsection{The Peaceful Transfer of Power}
            \label{gm--transfer}
            \begin{enumerate}
                \item Upon completion of election at an AGM, the newly elected take upon their title, with the inclusion of ‘elect’. For example, President Elect.
                
                \item The current executive shall remain in power until the 5th week of the summer term, when the complete transfer of power shall become valid, subject to conditions, where it would be more ideal to complete the transfer earlier.
                
                \item The current and elected executive should use the time prior to the complete transfer, as a handover and training period.
                
                \item Upon completion of an election at a Bye-Election, or a General Meeting triggered by impeachment, the newly elect will take upon their position and full power the position entails with immediate effect.
            \end{enumerate}
            
    \clearpage
    \section{Bye-Elections}
        \label{bye-election}
        \begin{enumerate}
            \item Any proceeding which may occur at a Bye-Election may also occur at a General Meeting (\ref{gm}).
        \end{enumerate}

        \subsection{Election Procedure}
            \label{bye-election--election-procedure}
            \begin{enumerate}
                \item The execution of all Society Elections shall be the responsibility of the Returning officer, who shall be the current President, as well as a neutral individual who is not a member of the society.
                    \subitem In the event that the President declares candidacy for the same elections, the Vice President shall instead be the Returning Officer.
                    
                    \subitem In the event that the Vice President also declares candidacy for the same elections, a suitably competent and independent member of the Cabinet shall be nominated for the role.
                    
                \item The Membership shall be informed as to the date of the election, available positions, and instruction as to declaration of candidacy at least four (4) weeks prior to the election.
                
                \item ”Re-Open Nominations (RON)” shall be an option for all positions in all elections.
                
                \item Preceding elections shall be speeches from all candidates, followed by questions put to the candidates by the Assembly.
                    \subitem The maximum length of speeches shall be determined in advance by the Returning Officer.
                
                \item In the event a candidate is unable to attend, or present, the candidate can provide written consent to the Returning Officer to authorise an individual to give the speech and answer questions on their behalf, provided the individual is not banned.
                    \subitem This Individual authorised may be The Returning Officer themselves, who will always be available for the role.
                
                \item In the event that a candidate is absent, without forewarning the Returning Officer, that candidate shall be excluded from the election.
                
                \item Each member of the Society shall be entitled to exactly one (1) vote per position.
                
                \item All members of the Society shall have the right to cast said vote either in-person or by proxy.
                
                \item Votes to be made by proxy shall be delegated to another member of the Society who must be present at the election.
                
                \item The Returning Officer shall be jointly responsible for the counting of the ballot, along with a neutral individual who is not a member of the society.
                
                \item Complaints regarding the elections should be brought before the Returning Officer.
                
                \item In the event of unsatisfactory resolution, complaints shall be escalated to the Student Union.
            \end{enumerate}
            
    \clearpage
    \section{Finance}
        \label{finance}
        \begin{enumerate}
            \item All matters pertaining to Society finance shall be carried out in accordance with the Financial Regulations of the Union and all applicable laws.
            
            \item Prior to the conclusion of the Academic Year, the Treasurer shall issue a Treasurer's Report.
                \subitem This Report shall be presented to the Assembly during the Annual General Meeting.
                \subitem This Report shall be displayed so as to be accessible to all members at any point during the following year.
            \item Any relevant documents which need to be archived should be done in compliance with the General Data Protection Regulation.
            
            \item Signatories to the Executive Account with the Union or any other applicable financial institution shall ordinarily be:
                \begin{enumerate}
                    \item the President 
                    \item the Vice-President
                    \item the Treasurer
                \end{enumerate}
        \end{enumerate}

    \clearpage
    \section{Projects}
        \label{project}
        \begin{enumerate}
            \item At any point, the Executive may, by simple majority vote, initiate a Project.

            \item Each Project shall be assigned a Technical Director to manage and oversee it.

            \item A project may be suspended, defunct or restored at the discretion of the executive.
        \end{enumerate}

    \clearpage
    \section{Affiliation and Sponsorship}
        \label{affiliation}
        \begin{enumerate}
            \item The society may engage with external bodies in compliance with Section of \href{https://lancastersu.co.uk/resources/articles-of-association-2023/download_attachment}{The Constitution of the Union, Article 47, Section 4}.
        \end{enumerate}
    \clearpage
    \section{Complaints and Appeals procedure}
        \label{complaints-appeals}
        \begin{enumerate}
            \item Complaints and Appeals should first be presented to the Well-being Officer, who will in turn share this information with the executive.
            \item In the event of an unsatisfactory resolution, complaints should be escalated by the complainant directly to the President.
            \item In the event of an unsatisfactory resolution, complaints shall be escalated to the Student Union.
            \item Complaints regarding elections should be brought before the Returning Officer.
            \item In the event of an unsatisfactory resolution regarding elections, complaints shall be escalated to the Student Union.
 
        \end{enumerate}
\end{document}
